\documentclass{article}

\usepackage{amsmath}
\usepackage{graphicx}
\usepackage{subcaption}


\author{Kalman Joseph Szenes}
\title{HPCSE2: Exercise 5}

\begin{document}
    \maketitle

    \section{Task 1}
    \subsection{a)}
    The peak performance is computed using the following formula: Clock Rate * \#Cores * \#Flop/s/Core = 0.00133 * 3584 = 4.763 [TFLOPS]


    The RAM bandwidth is computed using the following formula: Bus Width * Mem Clock Rate * 2 = 0.4096 / 8 * 715 = 732.16 [GB/s]

    \subsection{b)}
    Using the Roofline model, we can compute that the arithmetic intensity (AI) that would lead to ideal balance of ressources between memory and computation is around 6.5 FLOP/Byte.
    Anything under that is memory bound, anything over that is compute bound.
    The AI of DGEMM is given by the following formula: 
    \[
        \frac{m * n * k}{m * k + n * k + m * n}
    \] where M, N and K are the matrix dimensions.
    The AI that we obtain with our matrix dimensions is hence: 1638.4 [FLOP/Byte]. 
    This is well over the 6.5 [FLOP/Byte] and hence DGEMM is compute bound.

    






\end{document}